\documentclass{article}
\usepackage[margin=0.5in]{geometry}

\begin{document}
    \begin{itemize}
        \item Ram is used for long-term memory? False
        \item The clear() method empties a list and leaves you with []? True
        \item The append() method adds an element to the start of a list? False
        \item Python is an interpreted language: True
        \item Sequence[-1] returns the first element of a sequence. False
    \end{itemize}
    \section*{Short Answer}
    \begin{enumerate}
        \item What does the `CPU' do in a computer
        \begin{itemize}
            \item The `brain' of the computer
            \item Everything involves the cpu in one way or another.
            \item Processes Everything
            \item Responsible for running the code that we give it. (Only understands basic instructions in a basic language)
        \end{itemize}
        \item What does the `input()' function do in Python? What data type do you get back?
        \begin{itemize}
            \item Gets the user's input.
            \item Stores whatever they input as a \underline{string}
            \item Takes in a prompt to be displayed to the user.
        \end{itemize}
        \item Say we have some code like this:\\
        \begin{verbatim}
            num1 = 7**2
            num2 = 7 / 2
            num3 = 7 // 2
            print(num1, num2, num3)
        \end{verbatim}
        What will this print: `49 3.5 3'\\
        \item List 3 datatypes:
        \begin{itemize}
            \item String
            \item Float
            \item Integer
            \item Boolean
            \item List
            \item Tuple
            \item type
        \end{itemize}
        \item $[1, 2, 3] + [4, 5] = [1, 2, 3, 4, 5]$
        \item Range:
        \begin{itemize}
            \item range(10, 20) = [10, 11, 12, 13, 14, 15, 16, 17, 18, 19]
            \item range(50, 60, 2) = [50, 52, 54, 56, 58]
            \item range(-10, -25, -3) = [-10, -13, -16, -19, -22]
            \item range(20, 10, -1) = [20, 19, 18, 17, 16, 15, 14, 13, 12, 11]
        \end{itemize}
        \item 
\begin{verbatim}
list_sample = [1, 2, 3]
list_sample = list_sample * 2
list_sample.append(1)
print(list_sample) # [1, 2, 3, 1, 2, 3, 1]
list_sample.sort()
print(list_sample) # [1, 1, 1, 2, 2, 3, 3]
\end{verbatim}
    \item \begin{verbatim}
        my_string = "ThE BrOwN FoX RaN to The Farm"
        my_string.count("T") # 2
        my_string.lower() # "the brown fox ran to the farm"
        my_string.replace("a","z") # ThE BrOwN FoX RzN to The Fzrm
    \end{verbatim}
    \end{enumerate}

    \section*{Open Ended Questions}
    \begin{enumerate}
        \item In Computing, what do hardware and software do? Just a brief description is fine.\\
        Hardware is the components that make up the computer. Software is the coding.\\
        We need both hardware and software.\\
        \item What does it mean when we're taking a `slice' of a sequence (strings and lists)\\
        Taking a part of it.\\
        Slices return substrings or sublists\\
        Indexes return characters or individual items\\
        \item What are data types? What purpose do they serve?\\
        A way to categorize information within a program.\\
        It tells the program how it can interact with a given piece of data.\\
        Such as giving operators and methods.
        \item Define what a variable is.\\
        A `placeholder' value.\\
        A container that has data within it. The data can be of any data type, and has methods and operations that can be performed on it.
        \item What is mutable vs immutable?\\
        \textbf{Mutable} Data can be changed after assignment\\
        \textbf{Immutable} Data cannot be changed after assignment.
    \end{enumerate}
\end{document}